\documentclass[12pt,a4paper]{article}

\usepackage[UTF8]{ctex}
\usepackage{geometry}
\usepackage{fancyhdr}
\usepackage{enumitem}
\usepackage{xcolor}
\usepackage{titlesec}
\usepackage{hyperref}
\usepackage{amsmath,amsfonts,amssymb}
\usepackage{graphicx}
\usepackage{booktabs}

% 页面设置
\geometry{left=2.5cm,right=2.5cm,top=3cm,bottom=3cm}
\pagestyle{fancy}
\fancyhf{}
\fancyhead[L]{吴睿林 \ 科研周报}
\fancyhead[R]{\today}
\fancyfoot[C]{\thepage}

% 颜色定义
\definecolor{sectioncolor}{RGB}{51,102,153}
\definecolor{subsectioncolor}{RGB}{102,153,204}

% 标题格式
\titleformat{\section}{\Large\bfseries\color{sectioncolor}}{}{0em}{}[\titlerule]
\titleformat{\subsection}{\large\bfseries\color{subsectioncolor}}{}{0em}{}

% 列表格式
\setlist[itemize]{leftmargin=1.5em,itemsep=0.3em}
\setlist[enumerate]{leftmargin=1.5em,itemsep=0.3em}


\begin{document}

% 标题页
\begin{center}
    \vspace*{1cm}
        {\Large\bfseries 科研第一周: 基础铺垫与准备工作}\\[1cm]
    
    \begin{tabular}{ll}
        \textbf{日期:} & 2025 年第 21 周\\[0.3cm]
    \end{tabular}
\end{center}


\section{本周工作总结}

\subsection{学习内容}
\begin{itemize}
    \item \textbf{论文阅读}: 在浩华师兄的建议下, 开始阅读 AMP 算法入门文章 A Tutorial on AMP and State Evolution, 目前尚未完全读完, 在理解 condition lemma 上花了挺多时间
    \item \textbf{基础知识补充}: 为了更好理解 condition lemma 中的内容, 我补充学习了以下知识点并让 AI 针对性地写了专题讲义, 方便我今后系统性地复习, 
    讲义已上传到\notesref{}{我的 Github 仓库}:
    \begin{enumerate}
        \item \notesref{blob/main/study-notes/LinearAlgebra/KroneckerProduct/out/note.pdf}{Kronecker 积的定义、性质及其在矩阵运算中的应用}
        \begin{itemize}
            \item 在将矩阵向量化时 Kronecker 积非常好用, 这是矩阵版本的最小二乘解的基础
        \end{itemize}
        \item \notesref{blob/main/study-notes/LinearAlgebra/LeastSquaresDecomp/out/note.pdf}{最小二乘解的理论基础和计算方法, 及其背后的矩阵分解}
        \begin{itemize}
            \item Tutorial 中的两个 condition lemma 中都是用到了矩阵分解和最小二乘解的工具, 这个专题讲义系统讲解了这部分工具. 其中投影算子的部分我需要加深理解
        \end{itemize}
        \item \notesref{blob/main/study-notes/Probability/MMSE/out/note.pdf}{MMSE 最小均方误差估计理论}
        \begin{itemize}
            \item 在该讲义中我系统学习了 MMSE 及其延伸的相关结论, 学习的动机来源于 AMP 迭代中的函数 $\eta_t$ 如果选取使得 State Evolution 中期望最小的函数, 那么对应于 MMSE
        \end{itemize}
    \end{enumerate}
    \item \textbf{课程回顾}: 回顾了\text{大数据分析中的算法}课程的主要内容, 这门课程广泛介绍了各种算法迭代格式, 但对算法数学性质的分析介绍得较少, 具体内容包括: 压缩感知, 低秩矩阵优化, 随机数值代数(随机矩阵乘法, 随机 SVD 分解算法), 最优运输, 次模优化, 整数规划, 随机优化方法(主要是随机梯度下降法, Adagrad, ADMM 等)

\end{itemize}

\subsection{科研工具配置}
\begin{itemize}
    \item \textbf{文献管理}: 配置了 Zotero 文献管理工具及相关插件, 实现以下功能:
    \begin{enumerate}
        \item 阅读时间记录
        \item 阅读进度跟踪
        \item 便捷翻译功能
    \end{enumerate}
    \item \textbf{学术笔记管理}: 在 Github 上创建了专门的学术笔记仓库, 用于上传和管理:
    \begin{enumerate}
        \item 每周周报
        \item 学习笔记和专题讲义
    \end{enumerate}
\end{itemize}



\section{下周大致安排}

\begin{enumerate}
    \item 继续理解 A Tutorial on AMP and State Evolution 中的证明细节
    \item 周一: 网球课期末考试,人工智能期末考试
    \item 周四: 户外探索期末考试,人工智能期末上机考试 
    \item 做完并行计算上机作业, 并行加速稀疏矩阵向量乘法
    \item 计划将学术笔记仓库链接更新至个人主页, 但个人主页还需进一步完善 (Claude Sonnet 4 写代码真的好厉害!)
\end{enumerate}


\vspace{1cm}
\noindent\rule{\textwidth}{0.5pt}
\begin{center}
\textit{本周报共 \pageref{LastPage} 页}
\end{center}

\end{document}