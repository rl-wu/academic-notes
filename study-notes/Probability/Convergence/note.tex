\documentclass[12pt]{article}
\usepackage{amsmath, amssymb, amsthm}
\usepackage[UTF8]{ctex}
\usepackage{geometry}
\geometry{a4paper, margin=1in}

\newtheorem{definition}{定义}
\newtheorem{theorem}{定理}
\newtheorem{proposition}{命题}
\newtheorem{lemma}{引理}
\newtheorem{corollary}{推论}

\begin{document}

\title{概率统计中的收敛}
\author{}
\date{}
\maketitle

\section{概率收敛的基本概念}

\begin{definition}[几乎必然收敛]
设 $\{X_n\}_{n=1}^{\infty}$ 是定义在概率空间 $(\Omega, \mathcal{F}, P)$ 上的随机变量序列,$X$ 是随机变量。称 $X_n$ \textbf{几乎必然收敛}到 $X$,记作 $X_n \xrightarrow{a.s.} X$,如果
$$P\left(\lim_{n \to \infty} X_n = X\right) = 1$$
等价地,
$$P\left(\left\{\omega \in \Omega : \lim_{n \to \infty} X_n(\omega) = X(\omega)\right\}\right) = 1$$
\end{definition}

\begin{definition}[依概率收敛]
称 $X_n$ \textbf{依概率收敛}到 $X$,记作 $X_n \xrightarrow{P} X$,如果对任意 $\varepsilon > 0$,有
$$\lim_{n \to \infty} P(|X_n - X| > \varepsilon) = 0$$
\end{definition}

\begin{definition}[$L^p$ 收敛]
设 $p \geq 1$。称 $X_n$ 在 \textbf{$L^p$ 意义下收敛}到 $X$,记作 $X_n \xrightarrow{L^p} X$,如果
$$\lim_{n \to \infty} E[|X_n - X|^p] = 0$$
特别地,当 $p = 2$ 时称为\textbf{均方收敛},记作 $X_n \xrightarrow{m.s.} X$。
\end{definition}

\begin{definition}[依分布收敛/弱收敛]
称 $X_n$ \textbf{依分布收敛}到 $X$,记作 $X_n \xrightarrow{d} X$ 或 $X_n \Rightarrow X$,如果对 $X$ 的分布函数 $F$ 的所有连续点 $x$,有
$$\lim_{n \to \infty} F_n(x) = F(x)$$
其中 $F_n$ 是 $X_n$ 的分布函数。
\end{definition}

\section{收敛关系的层次结构}

\begin{theorem}[收敛的蕴含关系]
对于随机变量序列 $\{X_n\}$,有以下蕴含关系:
\begin{align}
X_n \xrightarrow{a.s.} X &\Rightarrow X_n \xrightarrow{P} X \\
X_n \xrightarrow{L^p} X &\Rightarrow X_n \xrightarrow{P} X \\
X_n \xrightarrow{P} X &\Rightarrow X_n \xrightarrow{d} X
\end{align}
特别地,几乎必然收敛和 $L^p$ 收敛之间没有一般的蕴含关系。
\end{theorem}

\section{弱收敛的重要性质}

\begin{theorem}[连续映射定理]
设 $g: \mathbb{R} \to \mathbb{R}$ 是连续函数,如果 $X_n \xrightarrow{d} X$,则
$$g(X_n) \xrightarrow{d} g(X)$$
\end{theorem}

\begin{theorem}[Slutsky定理]
设 $X_n \xrightarrow{d} X$,$Y_n \xrightarrow{P} c$(其中 $c$ 是常数),则:
\begin{align}
X_n + Y_n &\xrightarrow{d} X + c \\
X_n Y_n &\xrightarrow{d} cX \\
\frac{X_n}{Y_n} &\xrightarrow{d} \frac{X}{c} \quad (c \neq 0)
\end{align}
\end{theorem}

\begin{theorem}[有界连续函数的期望收敛]
设 $X_n \xrightarrow{d} X$,$h: \mathbb{R} \to \mathbb{R}$ 是有界连续函数,则
$$\lim_{n \to \infty} E[h(X_n)] = E[h(X)]$$
前提是相应的期望存在。
\end{theorem}

\section{大数定律}

\begin{theorem}[弱大数定律 (Khintchine)]
设 $\{X_n\}$ 是独立同分布的随机变量序列,$E[X_1] = \mu < \infty$。令 $\overline{X}_n = \frac{1}{n}\sum_{i=1}^n X_i$,则
$$\overline{X}_n \xrightarrow{P} \mu$$
\end{theorem}

\begin{theorem}[强大数定律 (Kolmogorov)]
设 $\{X_n\}$ 是独立同分布的随机变量序列,$E[|X_1|] < \infty$,$E[X_1] = \mu$。则
$$\overline{X}_n \xrightarrow{a.s.} \mu$$
\end{theorem}

\section{中心极限定理}

\begin{theorem}[经典中心极限定理 (Lindeberg-Lévy)]
设 $\{X_n\}$ 是独立同分布的随机变量序列,$E[X_1] = \mu$,$\text{Var}(X_1) = \sigma^2 \in (0, \infty)$。则
$$\frac{\sqrt{n}(\overline{X}_n - \mu)}{\sigma} \xrightarrow{d} N(0,1)$$
或等价地,
$$\sqrt{n}(\overline{X}_n - \mu) \xrightarrow{d} N(0, \sigma^2)$$
\end{theorem}

\begin{theorem}[Lindeberg中心极限定理]
设 $\{X_n\}$ 是独立随机变量序列,$E[X_k] = \mu_k$,$\text{Var}(X_k) = \sigma_k^2$。令
$$S_n = \sum_{k=1}^n X_k, \quad s_n^2 = \text{Var}(S_n) = \sum_{k=1}^n \sigma_k^2$$
如果 Lindeberg 条件成立:对任意 $\varepsilon > 0$,
$$\lim_{n \to \infty} \frac{1}{s_n^2} \sum_{k=1}^n E[(X_k - \mu_k)^2 \mathbf{1}_{|X_k - \mu_k| > \varepsilon s_n}] = 0$$
则
$$\frac{S_n - E[S_n]}{s_n} \xrightarrow{d} N(0,1)$$
\end{theorem}

\section{期望与极限的交换}

\begin{theorem}[单调收敛定理]
设 $\{X_n\}$ 是非负随机变量序列,且 $X_n \uparrow X$ a.s.(单调递增收敛),则
$$\lim_{n \to \infty} E[X_n] = E[X]$$
\end{theorem}

\begin{theorem}[Fatou引理]
设 $\{X_n\}$ 是非负随机变量序列,则
$$E[\liminf_{n \to \infty} X_n] \leq \liminf_{n \to \infty} E[X_n]$$
\end{theorem}

\begin{theorem}[控制收敛定理 (Lebesgue)]
设 $X_n \to X$ a.s.,且存在可积随机变量 $Y$ 使得 $|X_n| \leq Y$ a.s. 对所有 $n$ 成立,则
$$\lim_{n \to \infty} E[X_n] = E[X]$$
\end{theorem}

\section{弱收敛的等价刻画}

\begin{theorem}[Portmanteau定理]
对随机变量 $X_n$ 和 $X$,以下条件等价:
\begin{enumerate}
\item $X_n \xrightarrow{d} X$
\item 对所有有界连续函数 $f$:$E[f(X_n)] \to E[f(X)]$
\item 对所有闭集 $F$:$\limsup_{n \to \infty} P(X_n \in F) \leq P(X \in F)$
\item 对所有开集 $G$:$\liminf_{n \to \infty} P(X_n \in G) \geq P(X \in G)$
\item 对所有使得 $P(X \in \partial A) = 0$ 的Borel集 $A$:$P(X_n \in A) \to P(X \in A)$
\end{enumerate}
其中 $\partial A$ 表示集合 $A$ 的边界。
\end{theorem}

\section{多元情形}

\begin{definition}[多元弱收敛]
设 $\{X_n\}$ 是 $\mathbb{R}^d$ 上的随机向量序列。称 $X_n$ 弱收敛到 $X$,记作 $X_n \Rightarrow X$,如果对所有有界连续函数 $f: \mathbb{R}^d \to \mathbb{R}$,有
$$E[f(X_n)] \to E[f(X)]$$
\end{definition}

\begin{theorem}[多元中心极限定理]
设 $\{X_n\}$ 是 $\mathbb{R}^d$ 上独立同分布的随机向量序列,$E[X_1] = \mu$,$\text{Cov}(X_1) = \Sigma$(正定)。则
$$\sqrt{n}(\overline{X}_n - \mu) \xrightarrow{d} N(0, \Sigma)$$
\end{theorem}

\end{document}