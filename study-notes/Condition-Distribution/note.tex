\documentclass[11pt,a4paper]{ctexart}
\usepackage[utf8]{inputenc}
\usepackage[margin=1in]{geometry}
\usepackage{amsmath,amsthm,amssymb,amsfonts}
\usepackage{mathtools}
\usepackage{xcolor}
\usepackage[most]{tcolorbox}
\usepackage{enumitem}
\usepackage{fancyhdr}
\usepackage[colorlinks=true, linkcolor=blue]{hyperref}
\usepackage{tikz}
\usepackage{pgfplots}
\pgfplotsset{compat=1.16}

% 定理环境设置
\theoremstyle{definition}
\newtheorem{definition}{定义}[section]
\newtheorem{theorem}{定理}[section]
\newtheorem{lemma}{引理}[section]
\newtheorem{corollary}{推论}[section]
\newtheorem{proposition}{命题}[section]
\newtheorem{example}{例}[section]
\newtheorem{remark}{注}[section]

% 彩色框设置
\newtcolorbox{maintheorem}[1]{
    colback=blue!5!white,
    colframe=blue!75!black,
    title=#1,
    fonttitle=\bfseries,
    breakable
}

\newtcolorbox{keylemma}[1]{
    colback=red!5!white,
    colframe=red!75!black,
    title=#1,
    fonttitle=\bfseries,
    breakable
}

\newtcolorbox{importantdef}[1]{
    colback=green!5!white,
    colframe=green!75!black,
    title=#1,
    fonttitle=\bfseries,
    breakable
}

\newtcolorbox{insight}[1]{
    colback=orange!5!white,
    colframe=orange!75!black,
    title=#1,
    fonttitle=\bfseries,
    breakable
}

% 页眉页脚设置
\pagestyle{fancy}
\fancyhf{}
\fancyhead[L]{最小二乘理论与矩阵分解}
\fancyhead[R]{\thepage}
\renewcommand{\headrulewidth}{0.4pt}

% 数学符号定义
\newcommand{\RR}{\mathbb{R}}
\newcommand{\NN}{\mathbb{N}}
\newcommand{\EE}{\mathbb{E}}
\newcommand{\PP}{\mathbb{P}}
\newcommand{\Var}{\text{Var}}
\newcommand{\Cov}{\text{Cov}}
\newcommand{\argmin}{\text{argmin}}
\newcommand{\argmax}{\text{argmax}}
\newcommand{\trace}{\text{tr}}
\newcommand{\rank}{\text{rank}}
\newcommand{\Proj}{\text{Proj}}
\newcommand{\norm}[1]{\left\|#1\right\|}
\newcommand{\inner}[2]{\langle #1, #2 \rangle}
\newcommand{\given}{\,|\,}
\newcommand{\Range}{\text{Range}}
\newcommand{\Null}{\text{Null}}
\newcommand{\Col}{\text{Col}}
\newcommand{\Row}{\text{Row}}

\title{\textbf{最小二乘理论与矩阵分解专题讲义} \\ 
       \large 从经典理论到现代应用的系统性阐述}
\author{专题讲义}
\date{\today}

\begin{document}

\maketitle

\begin{abstract}
本讲义系统地介绍最小二乘理论及其相关的矩阵分解方法,从经典的向量最小二乘问题出发,逐步扩展到矩阵最小二乘、约束优化、以及在概率论中的应用。重点阐述矩阵分解(SVD、QR等)在解决这类问题中的核心作用,并展示这些工具如何应用于现代统计和信号处理问题中。
\end{abstract}

\tableofcontents
\newpage

\section{引言}

现代的最小二乘理论不仅包括经典的线性回归问题,还涵盖了矩阵优化、约束优化、正则化方法等多个方面。

矩阵分解技术,特别是奇异值分解(SVD)和QR分解,在解决最小二乘问题中发挥着核心作用。这些分解不仅提供了数值稳定的计算方法,更重要的是,它们揭示了问题的几何结构,使我们能够深入理解解的性质和存在性。

本讲义将系统地构建这一理论体系,从基础概念出发,逐步建立完整的理论框架。

\section{经典最小二乘理论}

\subsection{问题的提出}

\begin{importantdef}{最小二乘问题}
给定矩阵$A \in \RR^{m \times n}$和向量$\boldsymbol{b} \in \RR^m$,最小二乘问题是寻找向量$\boldsymbol{x} \in \RR^n$使得
$$\min_{\boldsymbol{x} \in \RR^n} \|A\boldsymbol{x} - \boldsymbol{b}\|_2^2$$
其中$\|\cdot\|_2$表示欧几里得范数。
\end{importantdef}

这个问题的几何意义是:在$A$的列空间$\Col(A) = \{A\boldsymbol{x} : \boldsymbol{x} \in \RR^n\}$中寻找最接近向量$\boldsymbol{b}$的点。

\subsection{几何解释}

最小二乘问题本质上是一个投影问题。设$\boldsymbol{x}^*$是最小二乘解,则$A\boldsymbol{x}^*$是$\boldsymbol{b}$在$\Col(A)$上的正交投影。

\begin{theorem}[最小二乘解的几何特征]
向量$\boldsymbol{x}^*$是最小二乘问题的解当且仅当残差向量$\boldsymbol{r} = \boldsymbol{b} - A\boldsymbol{x}^*$正交于$A$的列空间,即
$$A^T(\boldsymbol{b} - A\boldsymbol{x}^*) = \boldsymbol{0}$$
\end{theorem}

\begin{proof}
设$\boldsymbol{x}^*$为最小二乘解,对于任意$\boldsymbol{h} \in \RR^n$,定义函数
$$f(t) = \|A(\boldsymbol{x}^* + t\boldsymbol{h}) - \boldsymbol{b}\|_2^2$$

由于$\boldsymbol{x}^*$是最优解,$f(t)$在$t=0$处取得最小值,因此$f'(0) = 0$。

计算导数:
\begin{align}
f(t) &= \|A\boldsymbol{x}^* + tA\boldsymbol{h} - \boldsymbol{b}\|_2^2 \\
&= (\boldsymbol{r} + tA\boldsymbol{h})^T(\boldsymbol{r} + tA\boldsymbol{h}) \\
&= \|\boldsymbol{r}\|_2^2 + 2t\boldsymbol{r}^TA\boldsymbol{h} + t^2\|A\boldsymbol{h}\|_2^2
\end{align}

其中$\boldsymbol{r} = A\boldsymbol{x}^* - \boldsymbol{b}$。因此
$$f'(0) = 2\boldsymbol{r}^TA\boldsymbol{h} = 0$$

由于这对任意$\boldsymbol{h}$成立,得到$A^T\boldsymbol{r} = \boldsymbol{0}$。
\end{proof}

\subsection{正规方程}

从几何特征可以直接得到著名的正规方程:

\begin{maintheorem}{正规方程}
最小二乘问题的解满足正规方程:
$$A^TA\boldsymbol{x} = A^T\boldsymbol{b}$$
当$A^TA$可逆时,唯一解为:
$$\boldsymbol{x}^* = (A^TA)^{-1}A^T\boldsymbol{b}$$
\end{maintheorem}

\subsection{解的存在性和唯一性}

\begin{theorem}[最小二乘解的存在性和唯一性]
\begin{enumerate}[label=(\roman*)]
\item 最小二乘问题总是有解。
\item 当$A$的列线性无关(即$\rank(A) = n$)时,解唯一。
\item 当$A$的列线性相关时,解不唯一,但最小范数解唯一。
\end{enumerate}
\end{theorem}

\section{矩阵分解理论}

矩阵分解是解决最小二乘问题的核心工具。不同的分解方法揭示了矩阵的不同结构特征。

\subsection{奇异值分解(SVD)}

\begin{importantdef}{奇异值分解}
对于任意矩阵$A \in \RR^{m \times n}$,存在正交矩阵$U \in \RR^{m \times m}$和$V \in \RR^{n \times n}$,使得
$$A = U\Sigma V^T$$
其中$\Sigma \in \RR^{m \times n}$是对角矩阵,对角元素$\sigma_1 \geq \sigma_2 \geq \cdots \geq \sigma_r > 0$($r = \rank(A)$)称为奇异值。
\end{importantdef}

SVD的几何意义深刻:它将任意线性变换分解为三个基本变换的复合:旋转、缩放、再旋转。

\begin{insight}{SVD的几何意义}
SVD分解$A = U\Sigma V^T$可以理解为:
\begin{itemize}
\item $V^T$:在输入空间进行旋转,将输入向量对齐到主要方向
\item $\Sigma$:沿主要方向进行缩放,缩放因子为奇异值
\item $U$:在输出空间进行旋转,将结果对齐到输出空间的主要方向
\end{itemize}
\end{insight}

\subsection{Moore-Penrose伪逆}

当$A^TA$不可逆时,我们需要更一般的工具来求解最小二乘问题。

\begin{definition}[Moore-Penrose伪逆]
矩阵$A \in \RR^{m \times n}$的Moore-Penrose伪逆$A^+ \in \RR^{n \times m}$定义为满足以下四个Penrose条件的唯一矩阵:
\begin{enumerate}
\item $AA^+A = A$
\item $A^+AA^+ = A^+$
\item $(AA^+)^T = AA^+$
\item $(A^+A)^T = A^+A$
\end{enumerate}
\end{definition}

使用SVD可以显式构造伪逆:

\begin{theorem}[伪逆的SVD表示]
设$A = U\Sigma V^T$是$A$的SVD分解,则
$$A^+ = V\Sigma^+ U^T$$
其中$\Sigma^+ \in \RR^{n \times m}$定义为:当$\Sigma_{ii} \neq 0$时,$\Sigma^+_{ii} = 1/\Sigma_{ii}$;否则$\Sigma^+_{ii} = 0$。
\end{theorem}

\subsection{QR分解}

QR分解在数值计算中具有更好的稳定性。

\begin{definition}[QR分解]
对于列满秩矩阵$A \in \RR^{m \times n}$($m \geq n$),存在正交矩阵$Q \in \RR^{m \times m}$和上三角矩阵$R \in \RR^{n \times n}$,使得
$$A = Q\begin{bmatrix} R \\ 0 \end{bmatrix}$$
通常写作$A = Q_1R$,其中$Q_1 \in \RR^{m \times n}$是$Q$的前$n$列。
\end{definition}

\section{投影理论}

投影理论是理解最小二乘问题几何结构的关键。

\subsection{正交投影}

\begin{importantdef}{正交投影算子}
设$S \subseteq \RR^m$是子空间,正交投影算子$P_S: \RR^m \to S$定义为:对任意$\boldsymbol{v} \in \RR^m$,$P_S\boldsymbol{v}$是$S$中最接近$\boldsymbol{v}$的向量。
\end{importantdef}

\begin{theorem}[投影算子的性质]
正交投影算子$P_S$具有以下性质:
\begin{enumerate}[label=(\roman*)]
\item $P_S^2 = P_S$(幂等性)
\item $P_S^T = P_S$(自伴随性)
\item $\Range(P_S) = S$
\item $\Null(P_S) = S^\perp$
\item 对任意$\boldsymbol{v} \in \RR^m$,有$\boldsymbol{v} - P_S\boldsymbol{v} \perp S$
\end{enumerate}
\end{theorem}

\subsection{投影算子的矩阵表示}

\begin{theorem}[列空间投影算子]
设$A \in \RR^{m \times n}$,则$\boldsymbol{b}$在$\Col(A)$上的投影算子为:
$$P_{\Col(A)} = A(A^TA)^{-1}A^T \quad \text{(当$A$列满秩时)}$$
一般情况下:
$$P_{\Col(A)} = AA^+$$
\end{theorem}

\begin{theorem}[零空间投影算子]
零空间$\Null(A)$的投影算子为:
$$P_{\Null(A)} = I - A^+A$$
\end{theorem}

\section{广义最小二乘问题}

\subsection{约束最小二乘}

考虑带等式约束的最小二乘问题:
$$\min_{\boldsymbol{x}} \|A\boldsymbol{x} - \boldsymbol{b}\|_2^2 \quad \text{subject to} \quad C\boldsymbol{x} = \boldsymbol{d}$$

\begin{theorem}[约束最小二乘解]
约束最小二乘问题的解为:
$$\boldsymbol{x}^* = A^+\boldsymbol{b} + (I - A^+A)C^+({\boldsymbol{d} - CA^+\boldsymbol{b}})$$
其中第一项是无约束最小二乘解,第二项是约束修正项。
\end{theorem}

\subsection{加权最小二乘}

当观测具有不同精度时,使用加权最小二乘:
$$\min_{\boldsymbol{x}} \|W^{1/2}(A\boldsymbol{x} - \boldsymbol{b})\|_2^2 = \min_{\boldsymbol{x}} (A\boldsymbol{x} - \boldsymbol{b})^TW(A\boldsymbol{x} - \boldsymbol{b})$$

其中$W$是正定权重矩阵。

\begin{theorem}[加权最小二乘解]
加权最小二乘问题的解为:
$$\boldsymbol{x}^* = (A^TWA)^{-1}A^TW\boldsymbol{b}$$
\end{theorem}

\section{矩阵最小二乘问题}

现在我们将理论扩展到矩阵情况,这在现代应用中越来越重要。

\subsection{Frobenius范数最小化}

\begin{importantdef}{矩阵最小二乘问题}
给定矩阵$A \in \RR^{m \times p}$,$B \in \RR^{q \times n}$,$C \in \RR^{m \times n}$,矩阵最小二乘问题是寻找矩阵$X \in \RR^{p \times q}$使得
$$\min_{X} \|AXB - C\|_F^2$$
其中$\|\cdot\|_F$是Frobenius范数。
\end{importantdef}

\begin{theorem}[矩阵最小二乘解]
矩阵最小二乘问题$\min_X \|AXB - C\|_F^2$的解为:
$$X^* = A^+CB^+$$
当$A$和$B$分别列满秩和行满秩时,解为:
$$X^* = (A^TA)^{-1}A^TC B^T(BB^T)^{-1}$$
\end{theorem}

\begin{proof}
使用向量化技巧。注意到$\text{vec}(AXB) = (B^T \otimes A)\text{vec}(X)$,其中$\otimes$是Kronecker积。

因此原问题等价于:
$$\min_{\text{vec}(X)} \|(B^T \otimes A)\text{vec}(X) - \text{vec}(C)\|_2^2$$

这是标准的向量最小二乘问题,解为:
$$\text{vec}(X^*) = (B^T \otimes A)^+\text{vec}(C)$$

利用Kronecker积伪逆的性质$(B^T \otimes A)^+ = (B^T)^+ \otimes A^+ = B \otimes A^+$(当$A$和$B^T$列满秩时),得到
$$\text{vec}(X^*) = (B \otimes A^+)\text{vec}(C) = \text{vec}(A^+CB^+)$$

因此$X^* = A^+CB^+$。
\end{proof}

\subsection{多重约束的矩阵问题}

考虑同时满足左乘和右乘约束的问题:
$$\min_X \|X\|_F^2 \quad \text{subject to} \quad AX = C, \quad XB = D$$

这种问题在统计学中的条件分布分析中经常出现。

\begin{theorem}[双约束矩阵最小二乘]
问题$\min_X \|X\|_F^2$在约束$AX = C$和$XB = D$下的解(当约束相容时)为:
$$X^* = A^+C + D B^+ - A^+ADB^+$$
其中约束相容条件为$A^+CB = D$。
\end{theorem}

\section{矩阵分解在最小二乘中的核心作用}

\subsection{分解的统一观点}

所有主要的矩阵分解都可以理解为对特定几何结构的揭示:

\begin{insight}{矩阵分解的本质作用}
\begin{itemize}
\item \textbf{SVD}:揭示矩阵的完整几何结构,包括四个基本子空间的正交基
\item \textbf{QR分解}:将列空间的基正交化,保持数值稳定性
\item \textbf{特征值分解}:对称矩阵的主轴方向分析
\item \textbf{Cholesky分解}:正定矩阵的平方根分解
\end{itemize}

在最小二乘问题中,这些分解的作用是:
\begin{enumerate}
\item \textbf{降维}:将高维问题分解为低维子问题
\item \textbf{去耦}:将耦合的变量分离为独立的部分
\item \textbf{投影}:构造到各个子空间的投影算子
\item \textbf{稳定}:提供数值稳定的计算方法
\end{enumerate}
\end{insight}

\subsection{四个基本子空间}

对于矩阵$A \in \RR^{m \times n}$,SVD分解揭示了四个基本子空间:

\begin{theorem}[四个基本子空间]
设$A = U\Sigma V^T$是SVD分解,$r = \rank(A)$,则:
\begin{align}
\Col(A) &= \span\{u_1, u_2, \ldots, u_r\} \\
\Null(A^T) &= \span\{u_{r+1}, u_{r+2}, \ldots, u_m\} \\
\Row(A) &= \span\{v_1, v_2, \ldots, v_r\} \\
\Null(A) &= \span\{v_{r+1}, v_{r+2}, \ldots, v_n\}
\end{align}
其中$u_i$和$v_i$分别是$U$和$V$的列向量。
\end{theorem}

这四个子空间满足正交关系:$\Col(A) \perp \Null(A^T)$,$\Row(A) \perp \Null(A)$。

\subsection{投影的矩阵表示}

\begin{theorem}[SVD投影表示]
使用SVD分解,各子空间的投影算子为:
\begin{align}
P_{\Col(A)} &= \sum_{i=1}^r u_i u_i^T = U_r U_r^T \\
P_{\Null(A^T)} &= \sum_{i=r+1}^m u_i u_i^T = U_{m-r} U_{m-r}^T \\
P_{\Row(A)} &= \sum_{i=1}^r v_i v_i^T = V_r V_r^T \\
P_{\Null(A)} &= \sum_{i=r+1}^n v_i v_i^T = V_{n-r} V_{n-r}^T
\end{align}
其中$U_r, V_r$分别是$U, V$的前$r$列,$U_{m-r}, V_{n-r}$是剩余列。
\end{theorem}

\section{在概率论中的应用:条件分布构造}

\subsection{高斯分布的线性约束}

考虑高斯随机向量$\boldsymbol{z} \sim \mathcal{N}(\boldsymbol{0}, \sigma^2 I)$在线性约束$D\boldsymbol{z} = \boldsymbol{b}$下的条件分布。

这个问题的解决思路体现了最小二乘理论的完美应用:

\begin{keylemma}{条件分布的构造原理}
线性约束下高斯分布的条件分布可以分解为:
$$\boldsymbol{z} | D\boldsymbol{z} = \boldsymbol{b} = \text{(确定性最小范数解)} + \text{(投影到零空间的噪声)}$$

具体地:
$$\boldsymbol{z} | D\boldsymbol{z} = \boldsymbol{b} \stackrel{d}{=} D^+\boldsymbol{b} + P_{\Null(D)} \tilde{\boldsymbol{z}}$$
其中$\tilde{\boldsymbol{z}}$是$\boldsymbol{z}$的独立拷贝。
\end{keylemma}

\subsection{构造的几何直观}

这个构造的几何意义是:
\begin{enumerate}
\item $D^+\boldsymbol{b}$是满足约束$D\boldsymbol{z} = \boldsymbol{b}$的最小范数解
\item $P_{\Null(D)} \tilde{\boldsymbol{z}}$是投影到$D$零空间的独立高斯噪声
\item 两部分正交,因此可以独立处理
\end{enumerate}

\subsection{矩阵情况的推广}

对于矩阵约束问题:给定$A \sim \mathcal{N}(0, I)$(元素独立),约束$AQ = Y$和$M^TA = X^T$,条件分布为:
$$A | \text{约束} \stackrel{d}{=} E + \mathcal{P}(\tilde{A})$$

其中:
\begin{itemize}
\item $E$是满足约束的最小Frobenius范数解
\item $\mathcal{P}$是投影到约束零空间的投影算子
\item $\tilde{A}$是$A$的独立拷贝
\end{itemize}

最小范数解的构造使用了我们前面发展的理论:
$$E = \argmin\{\|A\|_F^2 : AQ = Y, A^TM = X\}$$

投影算子的构造为:
$$\mathcal{P}(A) = P_{M^\perp} A P_{Q^\perp}$$
其中$P_{M^\perp} = I - M(M^TM)^{-1}M^T$,$P_{Q^\perp} = I - Q(Q^TQ)^{-1}Q^T$。

\section{数值方法与计算考虑}

\subsection{条件数与数值稳定性}

\begin{definition}[条件数]
矩阵$A$的条件数定义为:
$$\kappa(A) = \|A\| \|A^+\| = \frac{\sigma_{\max}(A)}{\sigma_{\min}(A)}$$
其中$\sigma_{\max}, \sigma_{\min}$分别是最大和最小非零奇异值。
\end{definition}

当$\kappa(A)$很大时,最小二乘问题是病态的,数值解可能不可靠。

\subsection{正则化方法}

当问题病态时,常用正则化方法:

\begin{theorem}[Ridge回归]
Ridge回归问题
$$\min_{\boldsymbol{x}} \|A\boldsymbol{x} - \boldsymbol{b}\|_2^2 + \lambda\|\boldsymbol{x}\|_2^2$$
的解为:
$$\boldsymbol{x}_\lambda = (A^TA + \lambda I)^{-1}A^T\boldsymbol{b}$$
\end{theorem}

正则化的几何意义是在原解附近寻找范数较小的解,这在统计学中对应于引入先验信息。

\section{总结与展望}

\subsection{理论体系的统一性}

本讲义展示了最小二乘理论的统一性:
\begin{itemize}
\item \textbf{几何统一}:所有问题都可以理解为投影问题
\item \textbf{代数统一}:矩阵分解提供了统一的解析工具
\item \textbf{数值统一}:相同的数值方法适用于不同的问题变种
\end{itemize}

\subsection{矩阵分解的核心地位}

矩阵分解在最小二乘理论中的核心作用体现在:
\begin{enumerate}
\item \textbf{结构揭示}:分解揭示了问题的内在几何结构
\item \textbf{解的构造}:分解直接给出了解的显式表达式
\item \textbf{性质分析}:分解使我们能够分析解的存在性、唯一性等性质
\item \textbf{数值计算}:分解提供了数值稳定的计算方法
\end{enumerate}

\subsection{现代应用}

这些理论在现代统计学、机器学习、信号处理中有广泛应用:
\begin{itemize}
\item \textbf{高维统计}:稀疏回归、变量选择
\item \textbf{机器学习}:主成分分析、矩阵补全
\item \textbf{信号处理}:压缩感知、图像重构
\item \textbf{概率论}:条件分布、贝叶斯推断
\end{itemize}

理解这些基础理论对于掌握现代数据科学方法至关重要。

\begin{thebibliography}{99}
\bibitem{golub1996}
G. H. Golub and C. F. Van Loan,
\textit{Matrix Computations}, 3rd ed.
Johns Hopkins University Press, 1996.

\bibitem{trefethen1997}
L. N. Trefethen and D. Bau III,
\textit{Numerical Linear Algebra}.
SIAM, 1997.

\bibitem{horn2012}
R. A. Horn and C. R. Johnson,
\textit{Matrix Analysis}, 2nd ed.
Cambridge University Press, 2012.

\bibitem{bjorck1996}
A. Björck,
\textit{Numerical Methods for Least Squares Problems}.
SIAM, 1996.

\bibitem{stewart1998}
G. W. Stewart,
\textit{Matrix Algorithms Volume I: Basic Decompositions}.
SIAM, 1998.

\bibitem{watkins2010}
D. S. Watkins,
\textit{Fundamentals of Matrix Computations}, 3rd ed.
Wiley, 2010.
\end{thebibliography}

\end{document}