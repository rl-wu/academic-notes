\documentclass[12pt,a4paper]{article}

\usepackage[UTF8]{ctex}
\usepackage{geometry}
\usepackage{fancyhdr}
\usepackage{enumitem}
\usepackage{xcolor}
\usepackage{titlesec}
\usepackage{hyperref}
\usepackage{amsmath,amsfonts,amssymb}
\usepackage{graphicx}
\usepackage{booktabs}

% 页面设置
\geometry{left=2.5cm,right=2.5cm,top=3cm,bottom=3cm}
\pagestyle{fancy}
\fancyhf{}
\fancyhead[L]{吴睿林 \ 科研周报}
\fancyhead[R]{\today}
\fancyfoot[C]{\thepage}

% 颜色定义
\definecolor{sectioncolor}{RGB}{51,102,153}
\definecolor{subsectioncolor}{RGB}{102,153,204}

% 标题格式
\titleformat{\section}{\Large\bfseries\color{sectioncolor}}{}{0em}{}[\titlerule]
\titleformat{\subsection}{\large\bfseries\color{subsectioncolor}}{}{0em}{}

% 列表格式
\setlist[itemize]{leftmargin=1.5em,itemsep=0.3em}
\setlist[enumerate]{leftmargin=1.5em,itemsep=0.3em}


\newcommand{\R}{\mathbb{R}}
\newcommand{\N}{\mathbb{N}}
\newcommand{\norm}[1]{\|#1\|}
\newcommand{\abs}[1]{|#1|}
\newcommand{\inner}[2]{\langle #1, #2 \rangle}

% 定义数学操作符
\DeclareMathOperator{\Var}{Var}
\DeclareMathOperator{\Cov}{Cov}
\DeclareMathOperator{\Exp}{\mathbb{E}}

\begin{document}

% 标题页
\begin{center}
    \vspace*{1cm}
        {\Large\bfseries 第二周: 通识课期末考试 \& 理解高斯向量相关的结论}\\[1cm]
    
    \begin{tabular}{ll}
        \textbf{日期:} & 2025 年第 22 周\\[0.3cm]
    \end{tabular}
\end{center}


\section{本周工作总结}

\subsection{学习内容}
\begin{itemize}
    \item \textbf{期末考试}: 完成了 4 门通识课期末考试,接下来专业课考试和大作业提交日期集中在 6 月 17 日到 6 月 20 日
    
    \item \textbf{论文进度}: 继续阅读 The dynamics of message passing on dense graphs, with applications to compressed sensing,重点学习了第 3.7 节的几个核心引理
    \begin{itemize}
        \item 重点理解 3.7 节 Gaussian random vector 相关的引理
        \item 认识到 AMP 算法的理论结论在这里高度依赖于 Gaussian distribution 
    \end{itemize}
    
    \item \textbf{核心数学结论梳理}: 通过学习论文的第 3.7 节,重点理解了以下 3 个引理,这三个引理在文章中直接给出,没有给出证明:
    \[\boxed{
    P_W\,\bigl(\tilde{A}\,u\bigr) = D\,x,
    \quad\text{and}\quad
    \lim_{n\to\infty} \norm{x} = 0 
    \quad\text{almost surely.}}
    \]
    \[\boxed{
    \Exp\bigl[Z_1\,\varphi(Z_2)\bigr]
    =
    \Cov(Z_1,Z_2)\;\Exp\bigl[\varphi'(Z_2)\bigr].}
    \]
    \[\boxed{
    \Var\bigl(Z_t \mid Z_1,\dots,Z_{t-1}\bigr)
    =
    \Exp[Z_t^2]
    -u^{\mathsf{T}}\,C^{-1}\,u.}
    \]
    
    这些引理的详细描述和证明已整理并上传至 \notesref{blob/main/study-notes/Probability/LemmaNormalRandom/out/note.pdf}{GitHub 仓库的 LemmaNormalRandom 文件夹}
    
    \item \textbf{概率论基础复习}: 复习了概率中的各种收敛以及相关的结论
    \begin{enumerate}
        \item \textbf{弱收敛/依分布收敛}
        \item \textbf{依概率收敛}
        \item \textbf{几乎处处收敛}
    \end{enumerate}
    相关复习笔记已上传至 \notesref{blob/main/study-notes/Probability/Convergence/out/note.pdf}{GitHub 仓库的 Convergence 文件夹}

\end{itemize}



\section{下周大致安排}

\begin{enumerate}
    \item 继续阅读 \cite{bayati_dynamics_2011} 中的证明细节
    \item 完成数值分析课程的数值求解微分方程上机作业
    \item 完成并行计算上机作业: 实现 GPU 版本的 SafeSoftmax 算法
    \item 推进个人主页建设工作
\end{enumerate}

\vspace{1cm}
\noindent\rule{\textwidth}{0.5pt}


\begin{center}
\textit{本周报共 \pageref{LastPage} 页}
\end{center}

\end{document}